\documentclass[12pt,a4paper]{report}

% Pakete
\usepackage[utf8]{inputenc}
\usepackage[ngerman]{babel}
\usepackage{geometry}
\usepackage{fancyhdr}
\usepackage{titlesec}
\usepackage{setspace}
\usepackage{enumitem}
\usepackage{caption}
\usepackage{graphicx}
\usepackage{amsmath}
\usepackage{helvet} % Arial-ähnliche Schriftart
\usepackage{ragged2e}
\usepackage{listings}

% Schriftart auf Helvetica (Arial-ähnlich) setzen
\renewcommand{\familydefault}{\sfdefault}

% Seitenränder und Kopf-/Fußzeile
\geometry{
    top=2.5cm,
    bottom=2cm,
    left=2.5cm,
    right=2.5cm,
    headheight=1.25cm,
    headsep=0.5cm,
    footskip=1.25cm
}

% Zeilenabstand 1,3-fach
\setstretch{1.3}

% Blocksatz
\justifying

% Absatzabstand 12pt
\setlength{\parskip}{12pt}
\setlength{\parindent}{0pt}

% Kopf- und Fußzeile
\pagestyle{fancy}
\fancyhf{}
\fancyhead[L]{\leftmark}
\fancyhead[R]{\thepage}
\fancyhead[C]{}
\renewcommand{\headrulewidth}{0.4pt}
\renewcommand{\footrulewidth}{0pt}

% Kapitelmarkierungen für Kopfzeile
\renewcommand{\chaptermark}[1]{\markboth{#1}{}}

% Überschriftenformatierung
\titleformat{\chapter}[hang]
{\fontsize{16}{19.2}\bfseries\sffamily}
{\thechapter\hspace{1.75cm}}
{0pt}
{}
[\vspace{-6pt}]

\titleformat{\section}[hang]
{\fontsize{14}{16.8}\bfseries\sffamily}
{\thesection\hspace{1.75cm}}
{0pt}
{}
[\vspace{-6pt}]

\titleformat{\subsection}[hang]
{\fontsize{12}{14.4}\bfseries\sffamily}
{\thesubsection\hspace{1.75cm}}
{0pt}
{}
[\vspace{-6pt}]

\titleformat{\subsubsection}[hang]
{\fontsize{12}{14.4}\bfseries\sffamily}
{}
{0pt}
{}
[\vspace{-6pt}]

% Abstände vor und nach Überschriften
\titlespacing*{\chapter}{0pt}{18pt}{12pt}
\titlespacing*{\section}{0pt}{18pt}{6pt}
\titlespacing*{\subsection}{0pt}{18pt}{6pt}
\titlespacing*{\subsubsection}{0pt}{6pt}{6pt}

% Aufzählungen
\setlist[itemize]{
    leftmargin=*,
    itemsep=6pt,
    parsep=0pt,
    topsep=0pt,
    partopsep=0pt
}

% Beschriftungen
\captionsetup{
    font={size=\footnotesize,stretch=1,sf},
    justification=centering,
    skip=6pt,
    belowskip=18pt,
    size=\footnotesize
}

% Tabellenbeschriftung oben
\captionsetup[table]{position=top}

% Bildbeschriftung unten
\captionsetup[figure]{position=bottom}

% Nummerierung mit Kapitelnummer
\renewcommand{\thefigure}{\thechapter.\arabic{figure}}
\renewcommand{\thetable}{\thechapter.\arabic{table}}
\renewcommand{\theequation}{\thechapter.\arabic{equation}}

% Römische Seitenzahlen für Vorspann
\pagenumbering{roman}

\begin{document}

% Titelseite (Beispiel)
\begin{titlepage}
\centering
\vspace*{2cm}
{\huge\bfseries INF-1 PicoGo Praktikum Dokumentation}\\[2cm]
{\Large Bruce Weber}\\[1cm]
{\large \date{03.11.2025}}
\end{titlepage}

% Inhaltsverzeichnis
\tableofcontents
\newpage

% Arabische Seitenzahlen ab hier
\pagenumbering{arabic}

% Beispielinhalt
\chapter{Aufgabenstellung}
Dies ist die Einleitung der Arbeit.

\section{Materialliste}
So ein PICO GO set. Und noch paar folien auf moodle oder so. Aber ich glaube das wars

\section{Bereitgestellte Programme}
Voll krasse funktionalitäten wie $MotorControl.forward$\\
wow so packt man code in latex:
\begin{lstlisting}[language=Python, caption= Krasser Codeausschnitt mit Caption]
class MotorControl(object):
    ...
    def forward(self,speed):
            if((speed >= 0) and (speed <= 100)):
                self.PWMA.duty_u16(int(speed*0xFFFF/100))
                self.PWMB.duty_u16(int(speed*0xFFFF/100))
                self.AIN2.value(1)
                self.AIN1.value(0)
                self.BIN2.value(1)
                self.BIN1.value(0)
\end{lstlisting}
Die machen es uns Studenten so leicht die programme zu schreiben. Vielen Danke herr dr prof.

\chapter{Durchführung}
\section{Zusammenbauen}
Joah, kit halt zusammengesteckt, auf kosmetik hatten wir nicht wirklich bock

\section{Lösungsansätze}
\subsection{Objektorientierte Programmierung}

Voll krass, man nimmt so objekte aus realitäht, sagt: "Das (Klasse) ist (Instanz), hat das (Attribute), und kann sowas (Methode)".
So wie "Das Auto(Instanz) vom Modell ...(Klasse) hat einen Kofferraum (Attribut) und kann die KofferraumTür öffnen (Methode)

Modular, wenn seperation of concern gefolgt wird. Leicht erweiterbar ohne zu modifizieren.

Vorteile:
    Leicht verständlich, realer bezug
Nachteile:
    Oft speicherintensiv, nicht sehr effizient
    
\subsection{State Machine}
Es gibt zustände, wenn man in einem zustand ist, dann macht man erstmal etwas, und es gibt konditionen in denen sich der zustand ändert, dann ändert sich der zustand.

\subsection{Model View Control}
Model: Übernimmt logik
Control: Verbindet Model und View
View: Für den Nutzer

Nutzer für das Auto Program ist ein programmierer.
Auto funktionalitäten sollten leicht nutzbar sein, debuggen leicht, Seperation of Concern


\section{Lösung}

Github: https://github.com/BruceWeberStudHSNR/InfoPicoGo

Hier abschnitte von Code
View: AutoPicoGo Klasse

Control: Services, führen gewünschte feature aus

Modell ist in dem fall die Hardware, wird von COntrollern/Servicen angesteuert

\begin{figure}
    \centering
    \includegraphics[width=0.5\linewidth]{}
    \caption{Hier noch krasse Bild vom Auto}
    \label{fig:placeholder}
\end{figure}


% Beispiel für Aufzählung
\begin{itemize}
\item Erster Punkt,
\item Zweiter Punkt,
\item Dritter Punkt.
\end{itemize}

% Beispiel für Abbildung
\begin{figure}[h]
\centering
\rule{5cm}{3cm} % Platzhalter für Bild
\caption{Beispielabbildung}
\label{fig:beispiel}
\end{figure}

% Beispiel für Tabelle
\begin{table}[h]
\caption{Beispieltabelle}
\label{tab:beispiel}
\centering
\begin{tabular}{|c|c|}
\hline
Spalte 1 & Spalte 2 \\
\hline
Wert 1 & Wert 2 \\
\hline
\end{tabular}
\end{table}

% Beispiel für Formel
\begin{equation}
E = mc^2
\label{eq:einstein}
\end{equation}

\chapter{Zusammenfassung}
Wir haben so viel gelernt. 


\end{document}